\chapter{Conclusion}
\label{chap:conclusion}
To meet current and future application requirements, SoCs must integrate memory.  Designers have integrated to the point where over 90\% of the footprint is allocated to memory.  Because of the aggressive memory density and defects-per-million level, the overall SoC yield is strongly dependent on the health of the memory.  

Traditional circuit faults such as stuck-at and coupling must still be detected to maintain an acceptable yield.  A variety of memory test algorithms are implemented to screen for these faults with the March test becoming the most popular due to its mathematical backing and fast test time.  But because of the size, speed and complexity of today’s chips, running these tests through external memory testers is no longer a viable option and thus, MBISTs have become the only practical solution for fault coverage.  

This report shows that a programmable MBIST can achieve flexibility offered by external testers while also maintaining a low area overhead within the memory block.  An address generator, which offers Gray coding, 2\textsuperscript{\textit{i}}, and address complement counting methods, combined with an NPSF Type-1 neighborhood background pattern generator can be integrated with only an additional 5.2\% area for 8KB memories - and the area overhead is further reduced as memories increase in size.  This design will provide designers flexibility to adapt to new memory test models on the manufacturing line without sacrificing very much area.  

\section{Future Work}
\label{sect:bg-future}
This report offered two changes to the PMBIST design proposed in \cite{1584083}, but there are many other innovations in research now that will help make PMBIST a more robust solution.  Regarding the design offered in this report, there are a few other improvements that could help it to achieve better flexibility and more fault coverage.

\subsection{3-D Memory Testing}
This design proposed an NPSF pattern generator for 2-D memories where the memory layout is known.  With the advent of multi-layer memories and even 3-D memories, the coupling affects from the memory layers above and below the cell will need to be considered as transistor technology reaches the sub-10nm threshold.  The pattern generator could be improved upon by modifying the tile generator to output patterns above and below the base cell while still keeping the Type-1 neighborhood intact.

\subsection{Generate Additional Patterns}
The current design outputs the memory data values to generate a Type-1 NPSF tiling neighborhood in memory.  Additional patterns can be added to the generator create Type-2 NPSF neighborhoods.  Additionally, the two-group method could be employed for Type-1 neighborhoods instead of the tiling method to test the memory cells.

\subsection{Generate Additional Address Counting Methods}
The original design contained a linear and pseudo-random counting method and the proposed design extended that by adding Gray coding, Address Complement, and 2\textsuperscript{\textit{i}} address counting methods.  There are still other counting methods such as Worst-Case Gate Delay that can be added to the address generator that would improve the coverage and offer more flexibility to designers and testers.  

