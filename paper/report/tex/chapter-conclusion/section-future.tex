\section{Future Work}
\label{sect:bg-future}
This report offered two changes to the PMBIST design proposed in \cite{1584083}, but there are many other innovations in research now that will help make PMBIST a more robust solution.  Regarding the design offered in this report, there are a few other improvements that could help it to achieve better flexibility and more fault coverage.

\subsection{3-D Memory Testing}
This design proposed an NPSF pattern generator for 2-D memories where the memory layout is known.  With the advent of multi-layer memories and even 3-D memories, the coupling affects from the memory layers above and below the cell will need to be considered as transistor technology reaches the sub-10nm threshold.  The pattern generator could be improved upon by modifying the tile generator to output patterns above and below the base cell while still keeping the Type-1 neighborhood intact.

\subsection{Generate Additional Patterns}
The current design outputs the memory data values to generate a Type-1 NPSF tiling neighborhood in memory.  Additional patterns can be added to the generator create Type-2 NPSF neighborhoods.  Additionally, the two-group method could be employed for Type-1 neighborhoods instead of the tiling method to test the memory cells.

\subsection{Generate Additional Address Counting Methods}
The original design contained a linear and pseudo-random counting method and the proposed design extended that by adding Gray coding, Address Complement, and 2\textsuperscript{\textit{i}} address counting methods.  There are still other counting methods such as Worst-Case Gate Delay that can be added to the address generator that would improve the coverage and offer more flexibility to designers and testers.  
