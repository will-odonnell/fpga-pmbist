\chapter{First Stage SAR ADC}
The first stage SAR ADC is the lower precision of the two stages, so achieving appropriate accuracy is much simpler than for the second stage SAR ADC. The main goal for the first-stage SAR is to minimize the required conversion time, since the first stage shares both sampling and conversion in the same clock phase. To reduce sampling time, the capacitor array must be the minimum size to meet thermal noise requirements and the comparator must be as fast as possible.
\section{Capacitor Sizing}
Since a 4-bit resolution is initially assumed, the minimum capacitor sizes will not be limited by the thermal noise considerations of the SAR. Instead, the capacitor array will need to be sized so that the sampled voltage can meet the overall ADC resolution. From Equation \ref{eq:totalcapsize}, the total input capacitance must be 1.65 pF to match the quantization noise. This calculation, however, ignores the additional noise contributions from the MDAC op-amp and the second stage ADC. Assuming that half of the total noise contribution can come from the sampling capacitance, this increases the required capacitance to 3.2 pF. 
\section{Comparator Design}
The decision time of the first-stage comparator is the second largest limiting factor in the speed of this pipelined design. In order to achieve a fast decision time, a dynamic preamplifier-based comparator is used ~\cite{5714725}. This design uses two stages, the first being a preamplifier and the second being a latch. In order to avoid latching errors, a voltage of at least 20 mV on the latch must be maintained ~\cite{4541339}. For the 4-bit first stage, this won't affect the design of the preamplifier gain, but if the stage resolution is increased this limitation may need to be taken into account. Input referred offset noise will also need to be taken account in this comparator design. 