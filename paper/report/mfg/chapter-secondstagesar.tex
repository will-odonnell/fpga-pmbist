\chapter{Second Stage SAR ADC}
The second stage SAR ADC is the higher resolution stage, since it can partition the sample and conversion tasks into separate clock phases. Due to the high first stage closed loop gain, 8 in the initial design, input-referred noise will not be as much of a concern in the design of this stage. Since the first stage used a half-gain implementation, this requires the second stage ADC resolve down to 9 bits instead of the original 8 bits, making implementation more difficult. 
\section{Capacitor Sizing}
The main goal of capacitor sizing in the second stage is to match the quantization noise of a 9 bit ADC. Due to the high closed loop gain of the preceding stage, the input referred offset noise of the capacitor array will not be a significant contributor to overall noise. From Equation \ref{eq:totalcapsize}, the total input capacitance must be 26 fF to match the quantization noise. This value is too small to actually be able to implement with unit capacitors of reasonable matching, so thermal noise for this stage will not be a major concern. Assuming a C of 42 fF, the total capacitance of a 9 bit stage will be 10 pF. Alternatively, ~\cite{5714725} connects two unit capacitors in series to obtain an LSB capacitor of $\sfrac{1}{2}C$. This would reduce the theoretical capacitor array size to 5 pF. Even a load of 5 pF is significant in this design, so increasing the resolution of the first stage may be beneficial. 
\section{Comparator Design}
Since the second stage SAR has an entire clock phase to perform its conversion, a fully dynamic latch with no static power consumption and lower speed should be able to be used. In ~\cite{5282590}, a comparator that uses a latch specifically designed for low supply operation with no static power consumption and relatively high performance was demonstrated. This may be ideal for the implementation of the comparator in the second stage ADC. 