\chapter{Conclusion}
This chapter summarizes the results and discusses future work for this design.
\section{Discussion}
This work presented the initial design stages of a SAR-based pipeline ADC. With ever increasing focus on portable applications, power efficiency in ADCs will only become more important in the future. SAR ADCs are well known to be power efficient for applications requiring low sampling rates and medium to high resolutions. Applying the principles of pipelined ADCs to a SAR topology can increase the SAR application space to medium speed applications while still maintaining the power efficiency and high accuracy of the SAR topology. This design used a half-gain MDAC topology in order to reduce the required output swing and to increase overall ADC linearity. The half-gain topology alowed for the use of a triple-cascode OTA, which ended up being a low current solution. A novel scheme to obtain an additional bit of accuracy without increasing capacitance by utilizing the dummy LSB capacitor was also implemented. Although ideal clocks, comparators, and switches were used the OTA performance generally dominates the power and SNDR figures. The total figure of merit obtained from this design show that this topology has a lot of promise for power constrained designs.
\section{Future Work}
While this design covers the majority of the implementation of the ADC, there is still much to be done to ensure that this design would perform adequately when taped out. Some additional architectural modifications could reduce the load capacitance of the design signficantly. In addition, there are circuit level optimizations that could be performed on both the OTA and the digital control logic. Finally, some additional blocks need to be designed and additional simulations need to be performed. 

An extension of the implementation scheme in \ref{sec:dummylsbarch} could be used to obtain an additional bit of accuracy, or alternatively to halve the load capacitance in the second stage. Additionally, the differential scheme produces an additional bit of resolution that is currently not being utilized. This could lead to a factor of four reduction in load capacitance from the second stage capacitors. This could result in reduced power consumption or an increased sampling rate. These facts, along with realistic data on the parasitics from the OTA, suggest some re-architecting of the design could be done. The resolution of the two stages can be adjusted in order to achieve the best balance between sampling rate, power consumption, and total resolution. Furthermore, one large issue with the OTA design is that it will not scale well to future lower voltage processes. A reevaluation of the OTA topology used would like be a worthwhile exercise as well. Performing these architectural modifications could greatly improve the total performance of the design.

One disadvantage to the usage of the implementation scheme in \ref{sec:dummylsbarch} is the additional reference voltages required to obtain the additional bit of accuracy from the dummy LSB capacitor. The voltages used in this report were ideal, so the additional reference voltages did not affect the overall power consumption of the design. This would not be true in a real design, however. Designing additional voltage references that are at least 12-bit accurate is a non-trivial task and the implementation may end up consuming too much power to be beneficial for the design. An alternative scheme that introduces some imbalance in the SAR common-mode voltage can be used to obtain the additional bit without introducing more reference voltages. This scheme could only be implemented in the second stage of the pipeline, because introducing this imbalance could cause inaccuracies in the residue voltage. This is not a severe disadvantage, however, since the second stage sampling capacitance has a larger effect on the OTA power consumption than the first stage capacitance. Adding an additional bit to the first stage through traditional means would mean an increase in total area, but since the unit capacitance of the first stage is not at its minimum this increase could be mitigated slightly. This is a limitation that could likely be overcome. 

In addition to the architectural modifications the OTA and digital control logic performance could likely be improved at the circuit level. The OTA performance could likely be improved by reducing the input capacitance to the amplifier. Another area worth taking a second look at is the digital control logic. Although the OTA dominates total power consumption, the switching of the SAR capacitors and the digital control logic contribute about 20\% of the power. The digital logic was designed at the gate level using standard cells. It may be worthwhile to transform this design into register-transfer level (RTL) code, such as Verilog. Once this transformation is complete, a synthesis tool could be used to try to reduce the power consumption of the digital control logic even further. 

After completing this front-end work, the clocking network, comparators, and switches will need to be designed. Simulations will need to be run on all these blocks both individually and integrated into the ADC to ensure that the ADC will meet its performance requirements across all process corners. Once all this work is done, a layout and parasitic extraction can be performed. With this data, additional simulations and design adjustments can be performed in order to obtain a high level of confidence that a manufactured design will meet its specifications. Finally, the design will need to be manufactured and tested. Depending on the results from this testing, additional debugging and design adjustments may be necessary to achieve the desired performance. In the ideal case, first silicon would yield the desired results, but if this is not the case another manufacturing run could be performed with an additional round of testing to follow. If all goes well in this second stage, working silicon could be demonstrated.